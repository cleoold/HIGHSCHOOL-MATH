%%%%%%%%%%%%%%%%%%%%%%%%%%%%%%%%%%%%%%%%%%%%%%%%%%%%%
% summarised by cos. later information will be added soon. % 
%%%%%%%%%%%%%%%%%%%%%%%%%%%%%%%%%%%%%%%%%%%%%%%%%%%%%

\documentclass[10pt,landscape]{article}
\usepackage{amssymb,amsmath,amsthm,amsfonts}
\usepackage{multicol,multirow}
\usepackage{calc}
\usepackage{ifthen}
\usepackage[landscape]{geometry}
\usepackage[colorlinks=true,citecolor=blue,linkcolor=blue]{hyperref}
\usepackage{hyperref}
\usepackage{setspace}
\usepackage[utf8x]{inputenc}
\usepackage[english,russian]{babel}
\usepackage[para]{footmisc}
\usepackage{xcolor}
\usepackage{pagecolor,lipsum}
\usepackage{esint}
\usepackage{nicefrac}

\pagecolor{yellow!30!}

\ifthenelse{\lengthtest { \paperwidth = 11in}}
    { \geometry{top=.5in,left=.5in,right=.5in,bottom=.5in} }
	{\ifthenelse{ \lengthtest{ \paperwidth = 297mm}}
		{\geometry{top=1cm,left=1cm,right=1cm,bottom=1cm} }
		{\geometry{top=1cm,left=1cm,right=1cm,bottom=1cm} }
	}
\pagestyle{empty}
\makeatletter
\renewcommand{\section}{\@startsection{section}{1}{0mm}%
                                {-1ex plus -.5ex minus -.2ex}%
                                {0.5ex plus .2ex}%x
                                {\normalfont\large\bfseries}}
\renewcommand{\subsection}{\@startsection{subsection}{2}{0mm}%
                                {-1ex plus -.5ex minus -.2ex}%
                                {0.5ex plus .2ex}%
                                {\normalfont\normalsize\bfseries}}
\renewcommand{\subsubsection}{\@startsection{subsubsection}{3}{0mm}%
                                {-1ex plus 0ex minus .5ex}%
                                {0.5ex plus .2ex}%
                                {\normalfont\small\bfseries}}
                                
\newcommand{\spc}{\hspace*{1em}}
\renewcommand{\thefootnote}{(\arabic{footnote})}                                
\makeatother
\setcounter{secnumdepth}{0}
\setlength{\parindent}{0pt}
\setlength{\parskip}{0pt plus 0.5ex}
\onehalfspacing
% -----------------------------------------------------------------------

\title{}

\begin{document}

% \everymath{\displaystyle}
\raggedright
\footnotesize

\begin{center}
     \Large{\textbf{A formula summary for physics}} \\
\end{center}
\begin{multicols*}{3}
\setlength{\premulticols}{1pt}
\setlength{\postmulticols}{1pt}
\setlength{\multicolsep}{1pt}
\setlength{\columnsep}{2pt}

\section{Classical mechanics}

\subsection{Kinetics}
velocity: $\mathbf{v}=\frac{d\mathbf{r}}{dt}=r'_x \mathbf{i}+r'_y \mathbf{j}+r'_z \mathbf{k}$
\newline
acceleration: $\mathbf{a}=\frac{d\mathbf{v}}{dt}=v'_x \mathbf{i}+v'_y \mathbf{j}+v'_z \mathbf{k}$
\newline
motion laws:
\newline
\spc$v=v_0 +at$
\newline
\spc$x-x_0=v_0 t+ \frac{1}{2}at^2$
\newline
\spc$v^2-v_0^2=2a(x-x_0)$
\newline
\spc$x-x_0=\frac{(v_0+v)t}{2}$
\newline
centripetal acceleration: $a=\frac{v^2}{r}=\frac{4\pi ^2 r}{T^2}=4\pi ^2 rf^2=\omega ^2r$
\newline
velocity in two frames: $\mathbf{v}_{\mathrm{PA}}=\mathbf{v}_{\mathrm{PB}}+\mathbf{v}_{\mathrm{BA}}$
\newline
same acceleration measured in all frames: $\mathbf{a}_{\mathrm{PA}}=\mathbf{a}_{\mathrm{PB}}$

\subsection{Kinetics 2}
Newton's second law: $\mathbf{F}_\mathrm{T}=m\mathbf{a}$
\newline
Newton's third law: $\mathbf{F}_{\textrm{A to B}}=-\mathbf{F}_{\textrm{B to A}}$
\newline
\newline
acceleration of simple pulley: $a=\frac{m_1-m_2}{m_1+m_2}g$
\newline
\newline
drag force (body in fluid): $D=\frac{1}{2}C\rho Av^2$
\newline
\spc terminal speed: $v=\sqrt{\frac{2mg}{C\rho A}}$
\footnote{$C$: drag coefficient, $A$: effective cross-sectional area, $\rho$: air density.}
\newline
work: $W=\int \mathbf{F}\cdot d\mathbf{s}=\int_{x_i}^{x_f}F_x dx+\int_{y_i}^{y_f}F_y dy+\int_{z_i}^{z_f}F_z dz$
\newline
Hooke's law: $\mathbf{F}=-k\mathbf{x}$
\newline
\spc work by spring: $W=\frac{1}{2}kx_i^2-\frac{1}{2}kx_f^2$
\newline
kinetic energy: $K=\frac{1}{2}mv^2$
\newline
work-kinetic energy theorem: $W=K_f-K_i=\frac{1}{2}mv_f^2-\frac{1}{2}mv_i^2$
\newline
power: $P=\frac{dW}{dt}=\mathbf{F}\cdot \mathbf{v}$

\subsection{Kinetics 3}
definition of change in potential energy: $\Delta U=-W$
\newline
change in mechanical energy: $\Delta K+\Delta U=0$
\newline
total mechanical energy: $E=U+K$ (isolated)
\newline \newline
U-x graph:
\newline
external force: $F(x)=-\frac{d}{dx}U(x)$
\newline
neutral equilibrium: $E=U$
\newline
unstable equilibrium: $U''< 0, U'=0$
\newline
stable equilibrium: $U''> 0, U'=0$
\newline \newline
total m.energy of block-spring system: $E=\frac{1}{2}kx^2+\frac{1}{2}mv^2$
\newline
total m.energy of particle-earth system: $E=mgy+\frac{1}{2}mv^2$
\newline
conservation of energy: $\Delta K+\Delta U+\Delta E_{\textrm{internal}}\left ( + \textrm{other forms} \right )=0$ (isolated)
\newline
external work done: $W=\Delta K+\Delta U+\Delta E_{\textrm{internal}}$
\newline
change in energy: $\Delta E=\Delta K+\Delta U$
\newline
loss in mechanical energy (friction): $\Delta E=-fd$
\newline
energy loss due to emitted light: $E_x-E_y=hf$

\subsection{System kinetics}
centre of mass: $\mathbf{r}_{\mathrm{CM}}=\frac{1}{M} \sum _i m_i \mathbf{r}_i$
\newline
\spc continuous: $\mathbf{r}_{\mathrm{CM}}= \frac{1}{M}\int\mathbf{r}\rho dV= \frac{1}{V}\int\mathbf{r}dV$
\newline
\spc relation: $dm=\rho dV$
\newline \newline
linear momentum: $\mathbf{p}=m\mathbf{v}$
\newline
\spc relation: $K=\frac{p^2}{2m}$
\newline
net force: $\mathbf{F}_\mathrm{T}=\frac{d\mathbf{p}}{dt}$
\newline
Newton's second law: $\sum \mathbf{F}_{\textrm{external}}=M\mathbf{a}_{\mathrm{CM}}=\frac{d\mathbf{P}}{dt}$
\newline
conservation of linear momentum: $\mathbf{P}=\textrm{constant}$
\newline
\newline
циолковский's rocket formula:\newline
$\mathbf{F}=(\mathbf{v}-\mathbf{u})\frac{dm}{dt}+m\frac{d\mathbf{v}}{dt}$ or $\frac{d}{dt}m\mathbf{v}=\mathbf{F}+\mathbf{u}\frac{dm}{dt}$
\newline
\spc $v$: rocket's velocity in earth, $u$: fuel's velocity in earth
\newline \newline
change in translational k. energy: $\Delta K_{\mathrm{CM}}=F_{\mathrm{ext}}d_{\mathrm{CM}}$
\newline
König's theorem: $K=K_{\textrm{related to CM}}+\frac{1}{2}mv_{\mathrm{CM}}^2$

\subsection{Collisions}
impulse: $\mathbf{J}=\Delta \mathbf{p}=\int \mathbf{F}dt$
\newline
\newline
elastic collision:
\newline
\spc $v_1'=\frac{m_1-m_2}{m_1+m_2}v_1+\frac{2m_2}{m_1+m_2}v_2$
\newline
\spc $v_2'=\frac{m_2-m_1}{m_1+m_2}v_2+\frac{2m_1}{m_1+m_2}v_1$
\newline
\spc $v_\mathrm{CM}=\frac{P}{m_1+m_2}$
\newline \newline
complete inelastic collision: $m_1v_1+m_2v_2=(m_1+m_2)V_{\mathrm{CM}}$
\newline
heat of reactions: $Q=-\Delta mc^2$ ($+Q$: exothermic, $-Q$: endothermic)

\subsection{Rotation}
angular position: $\theta =s/r$
\newline
angular velocity: $\omega =\frac{d\theta}{dt}$
\newline
angular acceleration: $\alpha =\frac{d\omega}{dt}$
\newline
motion laws:
\newline
\spc $\omega =\omega _0+\alpha t$
\newline
\spc $\theta =\omega _0t+\frac{1}{2}\alpha t^2$
\newline
\spc $\omega ^2-\omega _0^2=2\alpha \theta $
\newline
\spc $\theta =\frac{(\omega _0+\omega )t}{2}$
\newline
linear-angular relation: $s=\theta r$, $v=\omega r$
\newline
\spc tangent acceleration: $a_t=\alpha r$
\newline
\spc radial acceleration: $a_r=\frac{v^2}{r}=\omega ^2r$
\newline \newline
rotational inertia: $I=\sum _i m_i r_i^2$
\newline
\spc r.i. for continuous objects: $I=\int r^2dm$
\newline
total rotational inertia: $I_{\mathrm{whole}}=\sum _i I_i$ (all to one axis)
\newline
parallel-axis theorem: $I=I_{\mathrm{CM}}+Mh^2$
\newline
perpendicular-axis theorem: $I_P=I_x+I_y$ (no thickness)
\newline
rotational kinetic energy: $K=\frac{1}{2}I\omega ^2$
\newline
torque: $\boldsymbol{\tau } =\mathbf{r}\times \mathbf{F}$
\newline
Newton's second law: $\tau _\mathrm{T}=I\alpha $
\newline
work: $W=\int F_t rd\theta =\int \tau d\theta $
\newline
power: $P=\frac{dW }{dt}=\tau \omega $
\newline
work-kinetic energy theorem: $W=\Delta K=\frac{1}{2}I\omega _f^2-\frac{1}{2}I\omega _i^2$

\subsubsection{Rotational inertia}
hoop, central axis: $I=MR^2$
\newline
hoop, diameter: $I=\frac{1}{2}MR^2$
\newline
annular cylinder, central axis: $\frac{1}{2}M\left ( R_1^2+R_2^2 \right )$
\newline
annular cylinder, central diameter,$\frac{1}{4}M\left ( R_1^2+R_2^2 \right )$
\newline
solid cylinder/disk, central axis: $\frac{1}{2}MR^2$
\newline
solid cylinder/disk, central diameter: $I=\frac{1}{4}MR^2+\frac{1}{12}ML^2$
\newline
rod, centre of length: $I=\frac{1}{12}ML^2$
\newline
rod, one end: $I=\frac{1}{3}ML^2$
\newline
triangle, parallel to base $a$ (whose height is $h$, through CM): $I=\frac{1}{18}Mh^2$
\newline
solid sphere, diameter: $I=\frac{2}{5}MR^2$
\newline
spherical shell, diameter: $I=\frac{2}{3}MR^2$
\newline
slab, centre: $I=\frac{1}{12}M(a^2+b^2)$
\newline
slab, along edge $b$:$I=\frac{1}{3}Ma^2$ 

\subsection{Rolling}
CM displacement/distance rolled: $x_{\mathrm{CM}}=\theta R$, $v_{\mathrm{CM}}=\omega R$
\newline
kinetic energy: $K=K_{\mathrm{rot}}+K_{\mathrm{tra}}=\frac{1}{2}I_{\mathrm{CM}}\omega ^2+\frac{1}{2}mv_{\mathrm{CM}}^2$
\newline
accleration of ideal yoyo: $a=-g(\frac{1}{1+I/MR^2})$
\newline
angular momentum: $\boldsymbol{\ell}=\mathbf{r}\times \mathbf{p}=\mathbf{r}\times m\mathbf{v}$
\newline
\spc a.m. for rigid, fixed axis: $L=I\omega $
\newline
angular impulse: $\Delta \boldsymbol{L}=\int \boldsymbol{\tau }dt$
\newline
Newton's second law: $\boldsymbol{\tau }_\mathrm{T}=\frac{d\boldsymbol{L}}{dt}$
\newline
conservation of angular momentum: $\boldsymbol{L}=\textrm{constant}$
\newpage

\subsection{Elasticity}
static equilibrium: $\mathbf{P}=0,\boldsymbol{L}=0$
\newline
requirements of equilibrium: $\sum \mathbf{F}_{\mathrm{ext}}=0,\sum \boldsymbol{\tau }_{\mathrm{ext}}=0$
\newline
tensile stress: $\frac{F}{A}=E\frac{\Delta L}{L}$, $E$: Young's modulus
\newline
sheering stress: $\frac{F}{A}=G\frac{\Delta x}{L}$, $G$: sheer modulus
\newline
hydraulic compression: $p=B\frac{\Delta V}{V}$, $B$: Bulk modulus

\subsection{Oscillation}
simple harmonic motion ($F=-m\omega^2x$):
\newline
\spc$\omega =\frac{2\pi }{T}=2\pi f$
\newline
\spc$x(t)=x_m \mathrm{cos}(\omega t+\phi )$
\newline
\spc$v(t)=-\omega x_m \mathrm{sin}(\omega t+\phi )$
\newline
\spc$a(t)=-\omega ^2 \, x(t)$
\newline \newline
linear oscillator
\newline
\spc definition: $\frac{d^2x}{dt^2}+\omega ^2x=0$
\newline
\spc angular frequency: $\omega =\sqrt{\frac{k}{m}}$
\newline
\spc period: $T=2 \pi \sqrt{\frac{m}{k}}$
\newline
\spc potential energy: $U(t)=\frac{1}{2}kx_m^2\mathrm{cos}^2(\omega t+ \phi)$
\newline
\spc kinetic energy: $K(t)=\frac{1}{2}kx_m^2\mathrm{sin}^2(\omega t+ \phi)$
\newline
\spc total energy: $E=\frac{1}{2}kx_m^2$
\newline
\spc series spring: $\frac{1}{K}=\sum _j \frac{1}{k_j}$, two: $K=\frac{k_1k_2}{k_1+k_2}$
\newline
\spc parallel spring: $K=\sum _j k_j$
\newline \newline
simple pendulum
\newline
\spc period: $T=2 \pi \sqrt{\frac{L}{g}}$
\newline
\spc restoring force: $F\approx -(\frac{mg}{L})s$
\newline \newline
torsion pendulum
\newline
\spc period: $T=2 \pi \sqrt{\frac{I}{\kappa }}$
\newline
\spc restoring torque: $\tau =-\kappa \theta $
\newline \newline
physical pendulum
\newline
\spc period: $T=2 \pi \sqrt{\frac{I}{mgh}}$
\newline
\spc restoring torque: $\tau =-(mg\mathrm{sin}\theta )h$
\newline \newline
damped simple harmonic motion
\newline
\spc damping force: $F_d=-bv$, $b$: damping constant
\newline
\spc definition: $\frac{d^2x}{dt^2}+\frac{b}{m}\frac{dx}{dt}+\frac{k}{m}x=0$
\newline
\spc \spc displacement: $x(t)=x_me^{-bt/2m}\mathrm{cos}(\omega _dt+ \phi)$
\newline
\spc angular frequency: $\omega_d=\sqrt{\frac{k}{m}-\frac{b^2}{4m^2}}$
\newline
\spc total energy: $E(t)\approx \frac{1}{2}kx_m^2e^{-bt/m}$

\subsection{Gravitation}
Newton's law of gravitation: $F=\frac{GMm}{r^2}$
\newline
gravitational constant: $G=6.67\cdot 10^{-11}\textrm{N}\cdot \textrm{m}^2/\textrm{kg}^2$
\newline
\spc differential: $dF=\frac{Gm_1}{r^2}dm$
\newline
gravitational field: $g=\frac{GM}{r^2}$
\newline
gravitational potential energy: $U=\int_{\infty }^{r}\frac{GMm}{x^2}dx=-\frac{GMm}{r}$
\newline
\spc escape speed: $v=\sqrt{\frac{2GM}{r}}$
\newline \newline
orbits:
\newline
\spc path of planet: $r=\frac{p}{1+e\mathrm{cos}\theta }$
\newline
\spc \spc $p=\frac{L^2}{GMm^2}$, $e=\sqrt{1+\frac{2EL^2}{G^2M^2m^3}}$
\newline
\spc net angular momentum: $L=mr^2\dot{\theta }$
\newline
\spc net mechanical energy: $E=\frac{1}{2}m(\dot{r}^2+r^2\dot{\theta} ^2)-\frac{GMm}{r}=\frac{1}{2}m\dot{r}^2+(\frac{L^2}{2mr^2}-\frac{GMm}{r})$
\newline
total energy of satellite-earth ellipse system:
$E=-K=-\frac{GMm}{2r}$
\newline
\spc perihelion: $r_1=a-c$
\newline
\spc aphelion: $r_2=a+c$
\newline
\spc\spc $a=\frac{r_1+r_2}{2}=-\frac{GMm}{2E}$, $c=\frac{r_2-r_1}{2}$, \spc\spc $b=\sqrt{a^2-c^2}=\sqrt{r_1r_2}=\frac{L}{\sqrt{-2mE}}$
\newline
total energy of satellite-earth hyperbola system: $E=\frac{GMm}{2r}$
\newline
total energy of satellite-earth parabola system: $E=0$
\newline
law of periods: $\frac{T^3}{r^3}=\frac{4 \pi ^2}{GM}$

\subsection{Fluids}
pressure: $p=\frac{\Delta F}{\Delta A}$ (all direction)
\newline
pressure in liquid: $p=p_0+\rho gh$
\newline
Pascal's principle: $\Delta p_{\textrm{int}}=\Delta p_{\textrm{ext}}$
\newline
Archimede's principle: $F_{\textrm{buoyancy}}=G_{\textrm{displaced water}}$
\newline
equation of continuity:
\newline
\spc volume flow rate $R=Av=\textrm{constant}$
\newline
\spc mass flow rate $m=Av\rho =\textrm{constant}$
\newline
Bernoulli's equation: $p+\frac{1}{2}\rho v^2+\rho gy=\textrm{constant}$

\subsection{Transverse waves}
transverse displacement: $y(x,t)=y_m\mathrm{sin}(kx-\omega t)$
\newline
\spc angular wave number: $k=\frac{2\pi }{\lambda }$
\newline
\spc waver number: $\kappa =\frac{1}{\lambda }$
\newline
\spc angular frequancy: $\omega =\frac{2 \pi}{T}$
\newline
\spc frequency: $f=\frac{1}{T}=\frac{\omega }{2 \pi}$
\newline
\spc wave speed: $v=\frac{\omega }{k}=\lambda f$
\newline
\spc \spc material expression: $v=\sqrt{\frac{\tau \textrm{(tension)}}{\mu (\textrm{density of media})}}$
\newline
\spc transverse speed: $u=\frac{\partial y}{\partial t}$
\newline
\spc average power: $\overline{P}=\frac{1}{2}\mu v\omega ^2y_m^2$
\newline \newline \newline \newline
adding waves
\newline
\spc superposition: $y=y_1+y_2=y_m\sin (kx-\omega t+\phi)+y_m\sin (kx-\omega t)$
\newline
\spc new wave: $y=(2y_m\mathrm{cos}\frac{1}{2}\phi)\mathrm{sin}(kx-\omega t+\frac{1}{2}\phi)$
\newline
\spc new amplitude: $2y_m\mathrm{cos}\frac{1}{2}\phi$
\newline
\spc phase shift: $+\frac{1}{2}\phi$
\newline
\newline
standing waves
\newline
\spc superposition: $y=y_1+y_2=y_m \sin (kx-\omega t)+y_m \sin (kx+\omega t)$
\newline
\spc new wave: $y=[(2y_m)\mathrm{sin}kx]\mathrm{cos}\omega t$
\newline
\spc new amplitude: $2y_m\mathrm{sin}kx$
\newline
\spc nodes: $x=n\frac{\lambda }{2}, n=0,1,2,...$
\newline
\spc antinodes: $x=(n+\frac{1}{2})\frac{\lambda }{2}, n=0,1,2,...$
\newline
\spc resonant frequency: $f_r=\frac{v}{\lambda }=\frac{v}{2l}n, n=1,2,3,...$

\subsection{Longitudinal waves}
speed of sound: $v=\sqrt{\frac{B}{\rho }}$
\newline
\spc bulk modulus: $B=-\frac{\Delta p}{\Delta V/V}(=\rho v^2)$
\newline
longitudinal displacement: $s=s_m\mathrm{cos}(kx- \omega t)$
\newline
air pressure: $\Delta p=\Delta p_m\mathrm{sin}(kx-\omega t)$
\newline
\spc relation: $\Delta p_m=(v\rho \omega )s_m$
\newline
\newline
interference
\newline
\spc phase shift: $\phi=\frac{\Delta d}{\lambda }2\pi$
\newline
\spc fully constructive: $\phi=m2\pi, m=0,1,2,...$
\newline
\spc fully destructive: $\phi=(m+\frac{1}{2})2\pi, m=0,1,2,...$
\newline
\newline
sound intensity: $I=\frac{1}{2}\rho v\omega ^2s_m^2$
\newline
sound level: $\beta =(10\textrm{ dB})\mathrm{log}(\frac{I}{I_0})$
\newline
\spc standard reference intensity: $I_0=10^{-12}\textrm{W}/\textrm{m}^2$
\newline \newline
resonant frequency
\newline
\spc pipe, two opens: $f_r=\frac{v}{\lambda }=\frac{v}{2L}n,n=1,2,3$
\newline
\spc pipe, one open: $f_r=\frac{v}{\lambda }=\frac{v}{4L}n,n=1,3,5,...$
\newline
beat frequency: $f_{beat}=f_1-f_2$
\newline
\newline
doppler effect: $f'=f\frac{v\pm v_L}{v\mp v_S}$
\newline
cone angle at supersonic speed: $\mathrm{sin}\theta =\frac{v}{v_s}$
\newpage
\section{Heat, Second law of thermodynamics}

\subsection{Heat}
coefficient of linear expansion: $\alpha =\frac{\Delta L/L}{\Delta T}$
\newline
\spc area expansion: $\beta =2\alpha $
\newline
\spc volume expansion: $\gamma =3\alpha $
\newline
heat capacity: $Q=cm(T_f-T_i)=C(T_f-T_i)$
\newline
heat of transformation: $Q=Lm$
\newline
volume work: $W=\int_{V_i}^{V_f}pdV$
\newline

first law of thermodynamics: $\Delta E_\textrm{int}=E_{\textrm{int,f}}-E_{\textrm{int,i}}=Q-W$
\newline
rate of heat transfer: $H=\frac{Q}{t}=kA\frac{T_H-T_C}{L}$\footnote{$k$: media's thermal conductivity.}
\newline
\spc multiple slabs: $H=A\frac{T_H-T_C}{\sum (L/k)}$

\subsection{Kinetic theory of gases}
ideal gas law: $pV=nRT$
\newline 
gas constant $R=8.31\textrm{J}/\textrm{mol}\cdot \textrm{K}$
\newline
volume work of expansion at constant pressure: $W=\int \frac{nRT}{V}dV=nRT \, \mathrm{ln}(\frac{V_f}{V_i})$
\newline
gas pressure: $p=\frac{nMv_{\textrm{rms}}^2}{3V}$
\newline
translational kinetic energy: $\overline{K}=\frac{3}{2}kT$
\newline 
Boltzman constant $k=R/N_A$
\newline
mean free path: $\lambda =\frac{1}{\sqrt{2}\pi dN/V}$\footnote{$d$: diameter, $N$: number of molecules.}
\newline
Maxwell's speed distribution: $P(v)=4\pi (\frac{M}{2 \pi RT})^{3/2}v^2e^{-Mv^2/2RT}$
\newline
\spc most propable speed: $v_p=\sqrt{\frac{2RT}{M}}$
\newline
\spc average speed: $\overline{v}=\sqrt{\frac{8RT}{\pi M}}$
\newline
\spc rms speed: $v_{\mathrm{rms}}=\sqrt{\frac{3RT}{M}}$
\newline
internal energy of monoatomic gas: $E_{\mathrm{int}}=(nN_A)\overline{K}=\frac{3}{2}nRT$
\newline
\spc monoatom: 3/2 ($f=1$)
\newline 
\spc diatom: 5/2 ($f=2$)
\newline
\spc 5-atom: 3 ($f=5$)
\newline
molar specific heat of monoatomic gas at constant volume: $C_v=\frac{3}{2}R=12.5\,\textrm{J/molK}$
\newline
\spc constant volume, change in internal energy: 
\newline
\spc $\Delta E_{\textrm{int}}=Q=nC_v(T_f-T_i)$
\newline
molar specific heat of monoatomic gas at constant pressure: $C_p-C_v=R$
\newline
\spc heat: $Q=nC_p(T_f-T_i)$
\newline
\spc work: $W=nR(T_f-T_i)$
\newline
law of adiabatic expansion: $pV^\gamma =\textrm{constant}$, or $TV^{\gamma -1}=\textrm{constant}$
\newline
\spc $\gamma =C_p/C_v=1+2/f$

\subsection{Second law of thermodynamics}
thermal effiency of engine: $e=\frac{|W|}{|Q_H|}=\frac{|Q_H|-|Q_C|}{|Q_H|}$
\newline
\spc max: $e_{\mathrm{Car}}=\frac{T_H-T_C}{T_H}$
\newline
coefficient of performance of refrigerator: $e=\frac{|Q_C|}{|W|}=\frac{|Q_C|}{|Q_H|-|Q_C|}$
\newline
\spc max: $e_{\mathrm{Car}}=\frac{T_C}{T_H-T_C}$
\newline
first law of thermodynamics in closed system: $|W|=|Q_H|-|Q_C|$
\newline
entropy: $dS=\frac{dQ}{T}$ and $∮ dS\leq 0$
\newline
reversible process: $S_f-S_i=\int _i^f dS=\int _i^f \frac{dQ}{T}$
\newline
free expansion: $S_f-S_i=\frac{1}{T}\int _i^f dQ=nR \, \mathrm{ln}\frac{V_f}{V_i}$
\newline
irreversible heat transfer: $S_f-S_i=cm\, \mathrm{ln}\frac{T^2}{T^2-\Delta T^2}$

\section{Electricity and Magnetism}
\subsection{Electrostatic forces}
Coulomb's law: $F=\frac{1}{4 \pi \epsilon _0}\frac{q_1q_2}{r^2}$
\newline
permitivity constant in vacuum: $\epsilon_0=8.85\cdot 10^{-12}\,\textrm{C}^{2}/\textrm{N}\cdot \textrm{m}^{2}$
\newline
charge is quantized: $q=ne$
\newline
\spc elementary charge $e=1.6\cdot 10^{-14}\,\textrm{C}$
\subsubsection{Electric field}
electric field: $\mathbf{E}=\frac{\mathbf{F}}{q_0}$
\newline
\spc differential: $d\mathbf{E}=\frac{1}{4\pi \epsilon _0}\frac{dq}{r^3}\mathbf{r}$ ($r$ from $dq$ to point)
\newline
\spc point charge: $E=\frac{1}{4 \pi\epsilon _0}\frac{q}{r^2}$
\newline
\spc straight rod (perpendicular): $E=\frac{\lambda a}{2\pi \epsilon _0 r}\frac{1}{\sqrt{4r^2+a^2}}$
\newline
\spc arc (to centre): $E=\frac{\lambda }{4 \pi \epsilon _0r}(2\mathrm{sin}\frac{\theta }{2})$
\newline
\spc ring (perpendicular): $E=\frac{qz}{4 \pi \epsilon _0(z^2+R^2)^{3/2}}$
\newline
\spc round disk (perpendicular): $E=\frac{\sigma }{2 \epsilon _0}(1-\frac{z}{\sqrt{z^2+R^2}})$
\newline
electrostatic force in a field: $\mathbf{F}=q\mathbf{E}$ (signed)
\subsubsection{Gauss' law}
Gauss' law: $\Phi _E=\oiint \mathbf{E}\cdot d\mathbf{S}=\frac{q}{\epsilon _0}$
\newline
\spc conducting surface: $E=\frac{\sigma }{\epsilon _0}$
\newline
\spc nonconducting surface: $E=\frac{\sigma }{2\epsilon _0}$
\newline
\spc straight rod: $\frac{\lambda }{2\pi r\epsilon _0}$
\newline
\spc two conducting plates (+ greater): $|E_L|=|E_R|=|E_{(+)}-E_{(-)}|$, $|E_{\textrm{in}}|=E_{(+)}+E_{(-)}=\frac{\sigma _1+\sigma _2}{\epsilon _0}$
\newline
\spc shell: $E=\frac{1}{4\pi \epsilon _0}\frac{q}{r^2}$ (outside), $E=0$ (inside)
\newline
\spc sphere: $E=\frac{1}{4\pi \epsilon _0}\frac{q}{r^2}$ (outside), $E=\left ( \frac{q}{4\pi \epsilon _0R^3} \right )r$ (inside)
\newline
\spc cylinder: $E=\frac{R^2\rho }{2\epsilon _0r}$ (outside), $E=\frac{\rho }{2\epsilon _0}r$ (inside)
\subsubsection{Potential}
work: $W=\int \mathbf{F}\cdot d\mathbf{s}=q_0\int \mathbf{E}\cdot d\mathbf{s}$
\newline
electric potential difference: $\Delta V=-\frac{W_{if}}{q_0}=\frac{\Delta U}{q_0}$
\newline
E-V relation: $\mathbf{E}=-\nabla V$, $V=-\int _{i_0}^f \mathbf{E}\cdot d\mathbf{s}$
\newline
electric field of parallel plates: $E=\frac{\Delta V}{\Delta d}$
\newline
point charge: $V=\frac{1}{4\pi \epsilon _0}\frac{q}{r}$(signed)
\newline
\spc discrete points: $V=\frac{1}{4\pi \epsilon _0}\sum _i\frac{q_i}{r_i}$
\newline
\spc continuous charge: $V=\frac{1}{4\pi \epsilon _0}\int \frac{dq}{r}$
\newline
\spc rod (perpendicular to one end): $V=\frac{\lambda }{4\pi \epsilon _0}\, \mathrm{ln}(\frac{L+(L^2+d^2)^{1/2}}{d})$
\newline
\spc arc (to centre): $V=\frac{\lambda \theta }{4\pi \epsilon _0}$
\newline
\spc ring (perpendicular): $V=\frac{q}{4\pi \epsilon_0\sqrt{z^2+R^2}}$
\newline
\spc disk (perpendicular): $V=\frac{\sigma }{2\epsilon _0}(\sqrt{z^2+R^2}-z)$

\subsection{Current and circuits}
current: $i=\frac{dq}{dt}$
\newline
current density: $J=i/A$
\newline
\spc relation: $i=\iint \mathbf{J}\cdot d\mathbf{A}$
\newline
\spc draft speed: $\mathbf{v}_\mathrm{d}=\mathbf{J}/(ne)$\footnote{ $n$: number of carriers per unit volume.}
\newline
resistance law: $R=\frac{V}{i}$
\newline
isotropic resistivity: $\rho=E/J$
\newline
\spc relation: $\mathbf{E}=\rho \mathbf{J}$
\newline
conductivity: $\sigma =1/\rho $
\newline
resistance: $R=\rho \frac{L}{A}$
\newline
\spc variation with temperature: $\rho -\rho _0=\rho _0\alpha (T-T_0)$\footnote{$\rho$: temperature coefficient of resistivity.}
\newline
\spc \spc$T_0=293\,\textrm{K}$, $\rho_0=1.69\,\mu\Omega\cdot\textrm{cm}$
\newline
rate of electricity supply: $P=iV$
\newline
\spc resistive dissipation: $P=i^2R=\frac{V^2}{R}$
\newline
electromotive force: $\varepsilon =\frac{dW}{dq}$
\newline
\spc supplying current: $i=\frac{\varepsilon }{R}$
\newline \newline
Kirchhoff's circuit laws
\newline
\spc resistance rule: $\Delta V=-iR$ (current), $\Delta V=+iR$ (opposite)
\newline
\spc emf rule: $\Delta V=+\varepsilon $ (current), $\Delta V=-\varepsilon $ (opposite)
\newline \newline
series charge: $q=q_1=q_2=...$
\newline
parallel charge: $q=\sum_j q_j$
\newline
series current: $i=i_1=i_2=...$
\newline
parallel current: $i=\sum_j i_j$
\newline
series voltage: $V=\sum _j V_j$
\newline
parallel voltage: $V=V_1=V_2=...$
\newline
series resistance: $R=\sum _j R_j$
\newline
parallel resistance: $\frac{1}{R}=\sum _j \frac{1}{R_j}$, two: $R=\frac{R_1R_2}{R_1+R_2}$
\newpage

\subsection{Capacitance}
capacitance: $C=\frac{q}{V}$
\newline
\spc parallel-plate: $C=\epsilon _0\frac{A}{d}$
\newline
\spc cylindrical: $C=2\pi \epsilon _0L\frac{1}{\mathrm{ln(b/a)}}$
\newline
\spc spherical: $C=4\pi \epsilon _0\frac{ab}{b-a}$
\newline
\spc isolated sphere: $C=4\pi \epsilon _0R$
\newline
series capacitor: $\frac{1}{C}=\sum_{j}\frac{1}{C_j}$
\newline
parallel capacitor: $C=\sum _j C_j$
\newline
potential energy: $U=\frac{q^2}{2C}=\frac{1}{2}CV^2$
\newline
volume energy density: $u=\frac{1}{2}\epsilon _0E^2$
\newline
\spc $q$ unchanged: $U_f=U_i/\kappa $
\newline
\spc $V$ unchanged: $U_f=\kappa U_i$
\newline \newline
RC circuit
\newline
\spc charging equation: $R\frac{dq}{dt}+\frac{q}{C}=\varepsilon $
\newline
\spc \spc charge function: $q=C\varepsilon (1-e^{-t/\tau _C })$
\newline
\spc \spc current function: $i=(\frac{\varepsilon }{R})e^{-t/\tau _C }$
\newline
\spc discharging equation: $R\frac{dq}{dt}+\frac{q}{C}=0 $
\newline
\spc \spc charge function: $q=q_0e^{-t/\tau _C }$
\newline
\spc \spc current function: $i=-i_0e^{-t/\tau _C }$
\newline
\spc capacitive time constant $\tau _C=RC$

\subsection{Magnetism}
force due to moving charge: $\mathbf{F}_B=q\mathbf{v}\times \mathbf{B}$
\newline
force due to current-carrying wire: $\mathbf{F}_B=i\mathbf{L}\times \mathbf{B}$ \newline
\spc $L$ along direction of conventional $i$
\newline
circular motion under $F_B$: $qvB=m\frac{v^2}{r}$
\newline
\spc period: $T=\frac{2\pi m}{qB}$
\newline
Hall effect, density of carriers: $n=\frac{Bi}{Vle}$
\newline
\spc $l=A/d$: thinkness of strip
\newline
Biot-Savart law: $d\mathbf{B}=\frac{\mu _0}{4\pi }\frac{id\mathbf{s\times \mathbf{r}}}{r^3}$
\newline
vacuum permeability: $\mu_0=4\pi\cdot10^{-7}\textrm{T}\cdot \textrm{m}/\textrm{A}$ (H/m)
\newline
\spc arc (to centre): $B=\frac{\mu _0i\theta }{4\pi R}$
\newline \newline
Ampere's circuital law: $\oint \mathbf{B}\cdot d\mathbf{s}=\mu _0i$
\newline
\spc long straight wire: $B=\frac{\mu _0i}{2\pi r}$
\newline
\spc solid wire: $B=\frac{\mu _0i}{2\pi r}$ (outside), $B=(\frac{\mu _0i}{2\pi R^2})r$ (inside)
\newline
\spc ideal solenoid: $B=\mu _0i_0n$, $n=N/L$: turns per unit length
\newline
\spc ideal toroid: $B=\frac{\mu _0i_0N}{2\pi r}$
\newline \newline
induced emf: $\varepsilon =-\frac{d\Phi _B}{dt}=-\frac{d}{dt} \iint \mathbf{B} \cdot d\mathbf{S}$
\newline
\spc for coils: $\varepsilon =-N\frac{d\Phi _B}{dt}$
\newline
Maxwell-Faraday equation: $\oint \mathbf{E}\cdot d\mathbf{s}=-\frac{d}{dt} \iint \mathbf{B}\cdot d\mathbf{S}$
\newline
\spc induced electrodynamic field, circle:
\\\spc$E=\frac{R^2}{2}\frac{dB}{dt}\frac{1}{r}$ (outside), $E=\frac{1}{2}\frac{dB}{dt}r$ (inside)

\subsection{Inductance}
inductance: $L=\frac{N\Phi _B}{i}$
\newline
\spc solenoid: $L/l=\mu _0n^2A$
\newline
\spc toroid: $L=\frac{\mu _0N^2h}{2\pi} \, \mathrm{ln}(\frac{b}{a})$
\newline
self-induced emf: $\varepsilon _L=-L\frac{di}{dt}$
\newline
potential energy: $U_B=\frac{1}{2}Li^2$
\newline
energy density: $u_B=\frac{B^2}{2\mu _0}$
\newline \newline
LR circuit
\newline
\spc rise in current: $iR+L\frac{di}{dt}=\varepsilon $
\newline
\spc\spc current function: $i=\frac{\varepsilon }{R}(1-e^{-t/\tau _L})$
\newline
\spc decay in current: $iR+L\frac{di}{dt}=0$
\newline
\spc\spc current function: $i=i_0e^{-t/\tau _L}$
\newline
\spc inductive time constant: $\tau _L=L/R$
\newline \newline
series inductance: $L=\sum _j L_j$
\newline
parallel inductance: $\frac{1}{L}=\sum _j \frac{1}{L_j}$
\newline
mutual induction, two coils: 
\newline
\spc $\varepsilon _2=-M\frac{di_1}{dt}$, $\varepsilon _1=-M\frac{di_2}{dt}$
\newline \newline
LC oscillation
\newline
\spc definition: $\frac{d^2q}{dt^2}+\frac{1}{LC}q=0$
\newline
\spc\spc charge function: $q=Q\mathrm{cos}(\omega t+\phi)$
\newline
\spc angular frequency: $\omega =\frac{1}{\sqrt{LC}}$
\newline
\spc electric potential energy: $U_E=\frac{Q^2}{2C}\mathrm{cos}^2(\omega t+\phi)$
\newline
\spc magnetic potential energy: $U_B=\frac{Q^2}{2C}\mathrm{sin}^2(\omega t+\phi)$
\newline
\spc total energy: $U=\frac{Q^2}{2C}$
\newline \newline
series RLC oscillation
\newline
\spc net energy dissipation: $\frac{dU}{dt}=-i^2R$
\newline
\spc definition: $\frac{d^2q}{dt^2}+\frac{1}{LR}\frac{dq}{dt}+\frac{1}{LC}q=0$
\newline
\spc \spc charge function: $q=Qe^{Rt/2L}\mathrm{cos}(\omega 't+\phi)$
\newline
\spc angular frequency: $\omega '=\sqrt{\omega ^2-(R/2L)^2}$

\subsection{Electromagnetic waves}
magnetic field induced by electric field: $\oint \mathbf{B}_E\cdot d\mathbf{s}=+\mu _0\epsilon _0\frac{d\Phi _E}{dt}=+\mu _0\epsilon _0\frac{d}{dt}\iint\mathbf{E}\cdot d\mathbf{S}$
\newline
"displacement current" between parallel plates, circle: 
\newline
\spc $B=\frac{\mu _0\epsilon _0R^2}{2}\frac{dE}{dt}\frac{1}{r}$ (outside), $\frac{\mu _0\epsilon _0}{2}\frac{dE}{dt}r$ (inside)
\newline
displacement current: $i_d=\epsilon _0\frac{d\Phi _E}{dt}$
\newline \newline
Electromagnetic waves
\newline
\spc B and E are in phase:
\newline
\spc\spc $E=E_m\mathrm{sin}(kx-\omega t)$
\newline
\spc\spc $B=B_m\mathrm{sin}(kx-\omega t)$
\newline
\spc wave speed: $c=\frac{\omega }{k}$
\newline
\spc magnitude ratio: $\frac{E_m}{B_m}=c$
\newline
\spc speed of light: $c=\frac{1}{\sqrt{\epsilon _0\mu _0}}$
\newline
\spc direction of wave/poynting vector: $\mathbf{S}=\frac{1}{\mu _0}\mathbf{E}\times \mathbf{B}$
\newline \newline
plane wave's instantaneous flow rate: $S=\frac{1}{c\mu _0}E^2$ ($S=P/A$)
\newline
wave intensity: $I=\overline{S}=\frac{1}{c\mu _0}E_{\textrm{rms}}^2$
\newline
momentum of light: 
\newline
\spc $\Delta p=\frac{\Delta U}{c}$ (total absorption), $\Delta p=\frac{2\Delta U}{c}$ (total reflection)
\newline
radiation pressure:
\newline
\spc $p_r=\frac{I}{c}$ (total absorption), $p_r=\frac{2I}{c}$ (total reflection)
\newline
law of Malus: $I=I_m \cos ^2\theta $

\subsection{AC}
resistive circuit: $V_R=I_RR$
\newline
capacitive circuit: $V_C=I_CX_C$
\newline
\spc capacitive reactance: $X_C=\frac{1}{\omega C}$
\newline
inductive circuit: $V_L=I_LX_L$
\newline
\spc inductive reactance: $X_L=\omega L$
\newline \newline
series RLC circuit
\newline
\spc current: $i=I\sin (\omega t-\phi)$
\newline
\spc voltage: $\varepsilon =v_R+v_C+v_L$
\newline
\spc current amplitude: $I=\frac{\varepsilon _m}{Z}$
\newline
\spc \spc impedance $Z=\sqrt{R^2+(X_L-X_C)^2}$
\newline
\spc phase constant: $\tan \phi=\frac{V_L-V_C}{V_R}=\frac{X_L-X_C}{R}$
\newline
\spc average power: $\overline{P}=I_{\textrm{rms}}^2R=\varepsilon _{\textrm{rms}}I_{\textrm{rms}}\cos \phi$
\newline
\spc $I$ is in phase with $v_R$; leads $v_C$ by 90°, lags hehind $v_L$ by 90°
\newline \newline
ideal transformer (rms)
\newline
\spc voltage: $\frac{V_s}{V_p}=\frac{N_s}{N_p}$ (AC supply at $p$ end, sends to $s$ end)
\newline
\spc current: $\frac{I_s}{I_p}=\frac{N_p}{N_s}$
\newline
\spc resistances: $R_{eq}=(\frac{N_p}{N_s})^2R$ ($R$ at $s$)

\subsection{Dipoles}
electric dipole
\newline
\spc electric field produced: $E=\frac{1}{2 \pi \epsilon _0}\frac{p}{z^3}$ (dipole axis)
\newline
\spc electric potential: $V(\theta)=\frac{1}{4\pi \epsilon _0}\frac{p\mathrm{cos}\theta }{r^2}$
\newline
\spc net torque: $\boldsymbol{\tau }=\mathbf{p}\times \mathbf{E}$
\newline
\spc potential energy: $U(\theta)=-\mathbf{p}\cdot \mathbf{E}$
\newline
\spc dipole moment: $\mathbf{p}=q\mathbf{d}$ ($-$ to $+$)
\newline \newline
magnetic dipole/current loop
\newline
\spc magnetic field produced: $\mathbf{B}=\frac{\mu _0}{2\pi }\frac{\boldsymbol{\mu }}{z^3}$
\newline
\spc net torque: $\boldsymbol{\tau }=\boldsymbol{\mu \times \mathbf{B}}$
\newline
\spc potential energy: $U(\theta)=-\boldsymbol{\mu \cdot \mathbf{B}}$
\newline
\spc magnetic dipole moment: $\boldsymbol{\mu }=Ni\mathbf{A}$, $N$: turns

\subsection{Maxwell's equations}
Gauss' law: $\oiint\mathbf{E}\cdot d\mathbf{S}=q/\epsilon _0$
\newline
Gauss' law: $\nabla\cdot \mathbf{E}=\rho /\epsilon _0$
\newline
Gauss' law for magnetism: $\oiint\mathbf{B}\cdot d\mathbf{S}=0$
\newline
Gauss' law for magnetism: $\nabla\cdot \mathbf{B}=0$
\newline
Maxwell-Faraday equation: $\oint \mathbf{E}\cdot d\mathbf{s}=-\frac{d}{dt} \iint \mathbf{B}\cdot d\mathbf{S}$
\newline
Maxwell-Faraday equation: $\nabla \times \mathbf{E}=-\frac{d\mathbf{B}}{dt}$
\newline
Ampere's circuital law: $\oint \mathbf{B}\cdot d\mathbf{s}=\mu _0i+\mu _0\epsilon _0\frac{d}{dt} \iint \mathbf{E}\cdot d\mathbf{S}$
\newline
Ampere's circuital law: $\nabla\times \mathbf{B}=\mu _0\mathbf{J}+\mu _0\epsilon _0\mathbf{E}$
\newline
electric displacement: $\mathbf{D}=\epsilon \mathbf{E}$
\newline
magnetic field: $\mathbf{H}=\mathbf{B}/\mu$

\subsection{Magnets}
magnetism due to spinning electron
\newline
\spc spin angular momentum: $S=\frac{h}{4 \pi}=5.2729\times 10^{-35} \textrm{J/s}$
\newline
\spc spin magnetic moment: $\mu _S=\mu _B=\frac{eh}{4 \pi m}=9.27\times 10^{-24}\textrm{J/T}$
\newline
magnetism due to orbital motion
\newline
\spc orbital angular momentum: $L_{\textrm{orb}}=mvr$
\newline
\spc orbital magnetic moment: $\mu _{\textrm{orb}}=\frac{1}{2}evr$
\newline
\spc \spc negative charge: $\boldsymbol{\mu }_{\textrm{orb}}=-\frac{e}{2m}\boldsymbol{L}_{\textrm{orb}}$

\section{Optics}

\subsection{Geometric optics}
law of reflection: $\theta _1=\theta _2$
\newline
law of refraction: $n_1\sin \theta _1=n_2\sin \theta _2$
\newline
total internal refraction, critical angle: $\theta _c=\sin^{-1}(\frac{n_2}{n_1})$ 
\newline
\spc (incident from greater $n_1$)
\newline
Brewster angle: $\theta =\tan^{-1}(\frac{n_2}{n_1})$ (incident from $n_1$)
\newline \newline
spherical mirror ($R$eal side is where reflected)
\newline
\spc focus: $f=\frac{r}{2}$ ($+$: concave, $-$: convex)
\newline
\spc relationship of object, image distance: $\frac{1}{p}+\frac{1}{i}=\frac{1}{f}$ 
\newline
\spc \spc($+$: $R$eal side, upright; $-$: $V$irtual side, inverted) ($p$ is $+$)
\newline
\spc lateral magnification: $|m|=\frac{h_{\textrm{image}}}{h_{\textrm{obj}}}$, $m=-\frac{i}{p}$ 
\newline
\spc \spc($+$: same orientation; $-$: opposite)
\newline
spherical refracting surface ($R$eal side is where refracted)
\newline
\spc relationship: $\frac{n_1}{p}+\frac{n_2}{i}=\frac{n_2-n_1}{r}$ ($p$ is $+$)
\newline \newline
thin lens
\newline
\spc relation 1: $\frac{1}{p}+\frac{1}{i}=\frac{1}{f}$
\newline
\spc relation 2: $\frac{1}{f}=(n-1)(\frac{1}{r_1}-\frac{1}{r_2})$
\newline
\spc \spc ($n=n_{\textrm{lens}}/n_{\textrm{medium}}$, $r_1$: first side light goes through)
\newline \newline
angular magnification, simple magnifer: $m_\theta =\frac{15\textrm{ cm}}{f}$
\newline
angular magnification, refracting telescope: $m_\theta =-\frac{f_{\textrm{ob}}}{f_{\textrm{eye}}}$
\newline
magnification, compound microscope: $M=-\frac{|f'_{\textrm{ob}}-f{\textrm{eye}}|}{f_{\textrm{ob}}}\frac{15\textrm{ cm}}{f_{\textrm{eye}}}$ 

\subsection{Interference and diffraction}
index of refraction: $n=\frac{c}{v}$
\newline
wavelength in medium: $\lambda _n=\frac{\lambda }{n}$
\newline
two mediums, same light, number of wavelength difference: 
\newline
\spc $N_2-N_1=\frac{L}{\lambda }(n_2-n_1)$
\newline \newline
double-slit interference
\newline
\spc fully constructive: $d\sin\theta =m\lambda,m=0,1,2,...$
\newline
\spc fully destructive: $d\sin\theta =(m+\frac{1}{2})\lambda,m=0,1,2,...$
\newline
\spc illumination intensity: $I=4I_0\cos ^2(\frac{1}{2}\phi)$, $\phi=\frac{2 \pi d}{\lambda} \sin \theta $ 
\newline
\spc\spc ($I_0$: intensity of one slit when the ohter covered, d: seperation of slits), $\overline{I}=2I_0$
\newline \newline
real double-slit
\newline
\spc intensity: $I=I_m\underbrace{(\cos ^2 \beta )}_{\textrm{intfr}}\underbrace{(\nicefrac{\sin \alpha}{\alpha)^2}}_{\textrm{diffr}}$
\newline
\spc \spc $\beta =(\frac{\pi d}{\lambda })\sin \theta $, $\alpha =(\frac{\pi a}{\lambda})\sin \theta $
\newline \newline
multiple slits ($N$ slits)
\newline
\spc grating maxima: $d\sin\theta =m\lambda,m=0,1,2,...$
\newline
\spc line width: $\Delta \theta =\frac{\lambda }{Nd\cos \theta }$
\newline
\spc dispersion/seperation of lines: $D=\frac{\Delta \theta }{\Delta \lambda }=\frac{m}{d\cos \theta }$
\newline
\spc resolving power: $R=Nm=\frac{\overline{\lambda} }{\Delta \lambda }$
\newline \newline
thin film, $n_1,n_3>n_2$ (incident at $n_1$)
\newline
(every larger $n$ of refraction side causes phase change of $\lambda/2$)
\newline
\spc fully constructive: $2n_2L=(m+\frac{1}{2})\lambda ,m=0,1,2,...$
\newline
\spc fully destructive: $2n_2L=m\lambda ,m=0,1,2,...$
\newline \newline
single-slit diffraction
\newline
\spc intensity minima: $a\sin \theta =m\lambda,m=1,2,3,...$ 
\newline
\spc intensity maximum: $I_m$ at centre
\newline
\spc intensity: $I=I_m(\frac{\sin \alpha }{\alpha })^2$, $\alpha =(\frac{\pi a }{\lambda })\sin \theta $ ($a$: width)
\newline \newline
Rayleigh's criteria: $\theta _R=1.22\frac{\lambda }{d}$ ($d$: lens' diameter)
\newline
\spc seperation of two sources: $\Delta x\approx f\theta $ ($f$: may be viewing distance)

\section{Modern physics}

\subsection{Special relativity}
speed parameter: $\beta =v/c$
\newline
Lorentz factor: $\gamma =\frac{1}{\sqrt{1-\beta ^2}}$
\newline
time dilation: $\Delta t =\gamma \Delta t _0$
\newline
length contraction: $L=\frac{L_0}{\gamma }$
\newline \newline
Lorentz transformation ($S,S'$):
\newline
\spc $\left\{\begin{array}{l}
x'=\gamma (x-vt) \\ t'=\gamma (t-vx/c^2)
\end{array}\right.$, $\left\{\begin{array}{l}
x=\gamma (x'+vt')\\ t=\gamma (t'+vx'/c^2)
\end{array}\right.$
\newline
difference in $x$, $t$:
\newline
\spc $\left\{\begin{array}{l}
\Delta x'=\gamma (\Delta x-v\Delta t)\\ \Delta t'=\gamma (\Delta t-v\Delta x/c^2)
\end{array}\right.$, $\left\{\begin{array}{l}
\Delta x=\gamma (\Delta x'+v\Delta t') \\ \Delta t=\gamma (\Delta t'+v\Delta x'/c^2)
\end{array}\right.$
\newline
relativistic velocity law: $V_{\textrm{CA}}=\frac{V_{\textrm{CB}}+V_{\textrm{BA}}}{1+V_{\textrm{CB}}V_{\textrm{BA}}/c^2}$
\newline
Doppler effect: $f=f_0\sqrt{\frac{1\pm \beta }{1\mp \beta }}$
\newline \newline
conservation of space-time interval: 
\newline
\spc $(\Delta s)^2=c^2(\Delta t)^2-(\Delta x)^2-(\Delta y)^2-(\Delta z)^2=\textrm{constant}$
\newline
relation of proper time: $\Delta s=c\Delta \tau$
\newline
4-displacement: $x^{\mu}(\tau)=[ct(\tau),x(\tau),y(\tau),z(\tau)]$
\newline
4-velocity: $u^{\mu}=\frac{d}{d\tau}x^{\mu}=\gamma(v)[c,v]$
\newline
\spc magniture of velocity: $|u^{\mu}|=\sqrt{(u^{\emph{0}})^2-(u^{\emph{1}})^2}=c$
\newline
accleration: $a^{\mu}=\frac{d}{d\tau}u^{\mu}\perp u^{\mu}$
\newline
4-momentum: $p^{\mu}=mu^{\mu}=m\gamma[c,v]=[\frac{E}{c},\gamma mv]$
\newline
relativistic kinetic energy: $K=mc^2(\gamma -1)$
\newline
\spc total energy: $E=\gamma mc^2=\underbrace{mc^2}_{\textrm{rest E}}+\underbrace{K}_{\textrm{kinetic E}}$
\newline
\spc relations: $E^2=(pc)^2+(mc^2)^2$, $(pc)^2=K^2+2Kmc^2$

\section{Quantum}

\subsection{Photons, particles}
single slit experiment: 
\newline
\spc distance between central and first max: $y=\frac{\lambda L}{d}$
\newline
\spc width of central max: $w=2y=\frac{2\lambda L}{d}$
\newline
De Braglie relation: $h=\lambda p=\lambda \sqrt{2mK}$
\newline
Planck's constant: $h=6.626\cdot 10^{-34}\,\mathrm{m}^2\cdot\textrm{kg}/\textrm{s}$
\newline
energy of photon: $E=cp=c\frac{h}{\lambda}=hf$
\newline
photoelectric effect: $K_{\mathrm{max}}=hf-\phi$
\newline
Compton scattering:
\newline
\spc electron stationary: $\lambda '-\lambda=\frac{h}{mc}(1-\cos\phi)$
\newline
\spc electron head-on: $\lambda '\approx \frac{hc}{E^e}\left [ 1+\frac{m_e^2c^4\lambda}{4hcE^e} \right ]$
\newpage
blackbody radiation (low frequency)
\newline
\spc distribution finding entities with energy $E$:
\newline
\spc \spc$p_E(E)=\frac{1}{KT}e^{-E/kT}$
\newline
\spc \spc average energy: $\overline{E}=kT$
\newline
\spc distribution of number of standing waves per unit volume:
\newline
\spc \spc $p_N(f)=\frac{N(f)}{V}=\frac{8\pi}{c^2}f^2kT$
\newline
\spc distribution of energy per unit volume
\newline
\spc \spc $p_{\nicefrac{E}{V}}(f)=\frac{E(f)}{V}=\frac{8\pi}{c^3}f^2kT$
\newline
\spc number of photons in standing wave: $n\approx \frac{kT}{hf}$
\newline
ultraviolet catastrophe (high frequency)
\newline
\spc energy of photons: $E=nhf$




\end{multicols*}
\end{document}
